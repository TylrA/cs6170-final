\documentclass[12pt, fullpage,letterpaper]{article}

% I copied older latex files that include packages we may need but idk,
% I'm not too knowledgeable about all things Latex
\usepackage[margin=1in]{geometry}
\usepackage{url}
\usepackage{amsmath}
\usepackage{amssymb}
\usepackage{xspace}
\usepackage{graphicx}
\usepackage{hyperref}
\usepackage{listings}
\usepackage{bm}

\title{Project Proposal}
\author{Tyler Adams: u0761872 \\Corbin Baldwin: u0292800}

\begin{document}
	\maketitle 
	\hrule 
	\vskip 0.5cm
	\section*{\normalfont Introduction}

	% Motivation and a brief project description.
	Rayleigh-Taylor instability (RTI) forms at the surface of contact between two fluids, one of which is denser and accelerates toward the other. Formations in RTI are not symmetric between the denser and lighter fluids, and the features of both sides serve to characterize the RTI. However, these features quickly rise in complexity, and it is unrealistic to simulate them beyond a given time threshold. In our project we hope to provide insights regarding structure that apply to RTI generally, including in the later stages of RTI development, using TDA methods.
  
	\section*{\normalfont Project Objective}  
%	What is the main objective/goal of your project?  What are you planning to
%	do to achieve your objective/goal?
	In this project, we plan to explore various topological filtrations of RTI data to determine 
	
	\section*{\normalfont Data} 
%	What are the type(s) of data your project will be dealing with?  How do you plan to get hold of such data sets?  What kind of insights are you planning to obtain from your data?
	We will collect our data via simulation. 
	
	\section*{\normalfont Background} 
%	What are the state-of-the-art techniques in dealing with the data of your interest?
	  
	\section*{\normalfont Technical Contributions} 
%	What are the expected technical contributions of your proposed work?
%	What are the differences and similarities between your proposed work and the state-of-the-art?
	
	\section*{\normalfont Expected Outcomes and Deliverables}  
%	What are the expected outcomes of your proposed project?
%	What do you plan to hand in?  (e.g.  source code, video demo, etc.)
	
	\section*{\normalfont Evaluation}  
%	What are the metrics to be used to evaluate how successful your project is once it is completed by the end of the semester?
	
	\section*{\normalfont Proposed Methods}  
%	What methods are you planning to use/develop?  What are your strategies in
%	tackling the proposed problem?

	\section*{\normalfont Software}  
%	What are the software (and possibly hardware) do you plan to use?  Or in the case you are working on software extensions, what is the baseline software you plan to work with?
	
	\section*{\normalfont Timelines}  
%	Between March 7th and April 23, what are the various milestones you plan to achieve along the way?

	Since we are attempting to reproduce the results of the paper, as well as do our own parametric analysis, we will closely follow their TDA pipeline. Our rough project timeline is as follows:
	\begin{enumerate}
		\item[{\textit{Week 1:}}] 
		Analyze the paper carefully and create software to generate a data-set for each time-step.
		\item[{\textit{Week 2:}}] 
		Learn how to extract isosurfaces at each time-step.
		\item[{\textit{Week 3:}}] 
		Extract and store a combinatorial Morse-Smale complex for each time-step.
		\item[{\textit{Week 4:}}] 
		Learn how to extract relevant homological information from the filtration necessary for the final step.
		\item[{\textit{Week 5:}}]  
		Construct merge-trees to display results.
		\item[{\textit{Week 6:}}] 
		Play around with parameters, prepare for project presentation, and begin work on final report. 
	\end{enumerate}  

	\section*{\normalfont Project Summary}  
%	Answer specific questions below using only 1-2 sentences:
%	\begin{enumerate}
%		\item What is an overview of your project?
%		
%		\item Why is the project worth pursuing?
%		
%		\item What are your project objectives?
%		
%		\item What are the questions you would like to answer?
%		
%		\item What data will you plan to use?
%		
%		\item How can we evaluate how successful your project is once it is completed?
%	\end{enumerate}

\end{document}
